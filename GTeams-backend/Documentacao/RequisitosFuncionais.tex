\documentclass[12pt]{article}
\usepackage[utf8]{inputenc}
\usepackage[brazil]{babel}
\usepackage{geometry}
\usepackage{enumitem}
\usepackage{titlesec}
\usepackage{hyperref}

\geometry{a4paper, margin=2.5cm}

\titleformat{\section}{\normalfont\Large\bfseries}{\thesection}{1em}{}

\title{Documentação Técnica \\ \large Sistema GTeams}
\author{Equipe de Desenvolvimento}
\date{\today}

\begin{document}

\maketitle

\section{Briefing Executivo}

O \textbf{GTeams} é um sistema de gestão de equipes e colaboradores voltado para organizações que precisam acompanhar, planejar e avaliar a atuação de seus times de maneira estruturada e eficiente.

A principal proposta do sistema é fornecer uma base de dados centralizada e segura onde gestores podem:

\begin{itemize}
    \item Cadastrar colaboradores e suas funções dentro de equipes;
    \item Estabelecer metas mensais e acompanhar a performance por intervalos de medição;
    \item Registrar observações relevantes sobre a atuação dos profissionais;
    \item Controlar datas personalizadas como férias, atestados, folgas e outras ocorrências;
    \item Visualizar todo o histórico do contrato de metas por colaborador.
\end{itemize}

O GTeams visa facilitar a tomada de decisão gerencial, oferecendo maior visibilidade sobre a produtividade das equipes, promovendo transparência e possibilitando ações corretivas em tempo hábil.

\textbf{Benefícios esperados:}
\begin{itemize}
    \item Centralização das informações de desempenho dos colaboradores;
    \item Otimização do planejamento de equipes e alocação de tarefas;
    \item Redução de falhas na gestão de metas;
    \item Facilidade de acesso a dados históricos e comparativos;
    \item Suporte à gestão estratégica com base em evidências.
\end{itemize}

\section{Especificação de Requisitos}

\subsection{Requisitos Funcionais}

\begin{enumerate}[label=RF\arabic*.]
    \item O sistema deve permitir o cadastro de \textbf{colaboradores}, incluindo nome, CPF, equipe associada e função (Operador, Administrador, Visualizador ou Supervisor).
    \item O sistema deve armazenar a \textbf{senha do colaborador} de forma segura utilizando criptografia baseada em salt e SHA512.
    \item O sistema deve permitir o registro de \textbf{equipes}, com múltiplos colaboradores associados.
    \item O sistema deve permitir o registro de \textbf{intervalos de medição}, com data inicial, data final e nome.
    \item O sistema deve associar metas mensais por equipe e intervalo de medição, armazenando metas para unidades NS e US.
    \item O sistema deve registrar \textbf{observações} feitas por colaboradores, associadas a um intervalo de datas.
    \item O sistema deve permitir o registro de \textbf{datas personalizadas} para colaboradores e medições, classificadas como: Útil, Fim de Semana, Férias, Atestado ou Meia Atuação.
    \item O sistema deve permitir o registro de \textbf{matrículas} dos colaboradores, com código e descrição.
\end{enumerate}

\subsection{Requisitos Não Funcionais}

\begin{enumerate}[label=RNF\arabic*.]
    \item O sistema deve ser desenvolvido utilizando a plataforma \texttt{.NET 8.0} com o Entity Framework Core.
    \item O sistema deve utilizar o padrão de projeto \texttt{Code First} para gerenciamento do banco.
    \item Os dados sensíveis devem ser armazenados com mecanismos de segurança baseados em hashing com salt e múltiplas iterações.
\end{enumerate}

\subsection{Modelo de Domínio}

As principais entidades do sistema incluem:
\begin{itemize}
    \item \textbf{Colaborador}
    \item \textbf{Equipe}
    \item \textbf{EquipeMetaMensal}
    \item \textbf{IntervaloMedicao}
    \item \textbf{Matricula}
    \item \textbf{Observacao}
    \item \textbf{DataPersonalizadaColaborador}
    \item \textbf{DataPersonalizadaMedicao}
\end{itemize}

\section{Considerações Finais}

O projeto GTeams está estruturado para permitir a evolução futura com suporte a múltiplos perfis de usuário, novas regras de negócio e integração com sistemas de relatório e dashboards. A adoção de boas práticas e tecnologias modernas garantem escalabilidade, segurança e clareza no modelo de dados.

\end{document}
